% Style commands go here.

% When using endnotes:
%   1) let footnotes in starred chapters undisturbed.
%   2) convert said footnotes into endnotes (rtfm).
\newif\ifendnotesStarredFinal
\endnotesStarredFinalfalse
% \endnotesStarredFinaltrue

% Choices: 1) use endnotes, but allow footnotes.
%          2) use only footnotes.
\newif\iffootnotesonly
\footnotesonlyfalse % 1
% \footnotesonlytrue %2

% Non-hyperlinked footnotes: a shortcut.
\newcommand\fn[1]{%
  \begin{NoHyper}%
    \footnote{#1}%
  \end{NoHyper}%
}

% If using only footnotes, use arabic. Otherwise, the endnotes will use arabic,
% so make the footnotes using Roman.
\iffootnotesonly
  \let\nn\fn
  \renewcommand*{\thefootnote}{\arabic{footnote}}
\else
  \newcommand{\nn}[1]{\endnote{#1}}
  \renewcommand*{\thefootnote}{\Roman{footnote}}

  % We're using endnotes, so setup its package.
  % --- Setup enotez package ---

% This version of the enotez setup works for appendices, but NOT for endnotes
% within starred chapters. In fact, until one has a final draft, in these
% chapters one should the \fn command to produce footnotes. For details how to
% setup endnotez for starred chapters once one does have a final draft, see
% https://randomwalk.eu/LaTeX/enotez-starred-and-appendix.

% To detect if a given (non-starred) chapter is, or not, an appendix.
\newif\ifinappendix% Default is \inappendixfalse

% Use endnotes for regular chapters, and footnotes for appendix. Also,
% \footnote{text} produces a footnote, using a symbol, to avoid confusion with
% endnotes.
\let\oldappendix\appendix% Store \appendix
\renewcommand{\appendix}{% Update \appendix
  \oldappendix% Default \appendix
  \inappendixtrue% Set switch to true
}

% For use in starred chapters.
% Make footnotes use symbols, and have no hyperlinks.
\renewcommand*{\thefootnote}{\fnsymbol{footnote}}
\newcommand\fn[1]{%
  \begin{NoHyper}%
    \footnote{#1}%
  \end{NoHyper}%
}

\setenotez{
  backref=true,
  totoc=chapter,
  counter-format=arabic,
  reset=true,
  split=chapter,
  split-title={<ref>. <title>}
}

% Grab the correct title, even for appendices.
\NewSplitTitleTag{title}{%
  \ifnum0<0<ref>\relax%
    \nameref*{ch:<split-level-id>}%
  \else%
    \nameref*{appdx:<ref>}%
  \fi%
}

% Requires package letltxmacro. Create a counter for numbered chapters, to be
% able to show chapter names in Notes' listing. The label is different for
% appendices, because for the these, the chapter counter is reset (if the label
% was the same, then chapter 1 and appendix A would have the same label, etc.).
\LetLtxMacro\origchapter\chapter
\RenewDocumentCommand\chapter{som}{%
  \IfBooleanTF{#1}
    {% starred chapter, no label then
      \origchapter*{#3}%
    }
    {% else add a label
      \IfNoValueTF{#2}
        {\origchapter{#3}}
        {\origchapter[#2]{#3}}%
        \ifinappendix%
          \expanded{\noexpand\label{appdx:\thechapter}}%
        \else%
          \expanded{\noexpand\label{ch:\thechapter}}%
        \fi%
    }%
}

% This is needed before setting up fancyhdr, to get the page header right in
% the Notes' chapter (this is only relevant when said chapter has more than one
% page...)
\setenotez{list-heading=\chapter*{#1\markboth{#1}{}}}

% Left-align start of note with the respective chapter heading.
\DeclareInstance{enotez-list}{plain}{paragraph}{format=\footnotesize\leftskip1.25em}


  \ifendnotesStarredFinal
    \let\fn\nn
  \fi
\fi

% Make the page a bit wider. Requires package geometry.
\geometry{top=2.5cm, bottom=2.5cm, left=3.0cm, right=3.0cm}

% Proper format of caption text.
% Requires package caption.
\captionsetup{font=small, labelfont=bf}

% To indicate current chapter name and number in the header.
% Requires package fancyhdr.
\pagestyle{fancy}
\cfoot{} % Get rid of the page number in the footer.
\makeatletter
\renewcommand{\chaptermark}[1]{\markboth{%
  \ifnum\c@secnumdepth > \m@ne \@chapapp\ \thechapter\ --\ \fi #1}{}}
\makeatother
\fancyhead[R]{}
\fancyhead[L]{\fontsize{10}{12}\selectfont\leftmark \hfill \thepage}

% Remove original page numbers (bottom, centered).
\cfoot{}
% Do the same for the page number in the chapter opening page (\cfoot{} leaves
% those..).
\fancypagestyle{plain}{%
  \renewcommand{\headrulewidth}{0pt}%
  \fancyhf{}%
}

% For fancy chapter header.
% Requires packages titlesec, blindtext, color.
\definecolor{gray75}{gray}{0.75}
\newcommand*{\hsp}{\hspace{20pt}}
\titleformat{\chapter}[hang]{\Huge\bfseries}{\thechapter\hsp\textcolor{gray75}{|}\hsp}{0pt}{\Huge\bfseries}

% To make the title footnote always use symbols.
% Note: with report class, this footnote always seem be without hyperlinks.
\let\origmaketitle\maketitle
\renewcommand*{\maketitle}{{%
  \renewcommand*{\thefootnote}{\fnsymbol{footnote}}%
  \origmaketitle}}

% To avoid fancy header in pages after the first, in TOC and bibliography.
\let\origtoc\tableofcontents
\renewcommand*{\tableofcontents}{%
  \hypersetup{linkcolor=black}%
  \origtoc\chaptermark{Contents}\cleardoublepage\hypersetup{linkcolor=MidnightBlue}}

\let\bibliographyorig\bibliography
\renewcommand*{\bibliography}[1]{%
  \bibliographyorig{#1}\chaptermark{References}\cleardoublepage}

% Uncomment when using an index. Makes the font small.
% \let\printindexorig\printindex
% \renewcommand*{\printindex}{%
%   \footnotesize\printindexorig\cleardoublepage}

% Proper format of caption text.
% Requires package caption.
\captionsetup{font=small, labelfont=bf}

% --- Reduce the spacing between items in environments that use items. ---
\let\oldenumerate\enumerate
\let\oldenumerateend\endenumerate
\renewenvironment{enumerate}{\oldenumerate\setlength{\itemsep}{0.25em} \setlength{\parskip}{0pt}}{\oldenumerateend}

\let\olditemize\itemize
\let\olditemizeend\enditemize
\renewenvironment{itemize}{\olditemize\setlength{\itemsep}{0.25em} \setlength{\parskip}{0pt}}{\olditemizeend}
% --- END Reduce the spacing between items in environments that use items. ---
