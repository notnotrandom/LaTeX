% Load ALL packages, in a single place. Package loading order matters A LOT,
% and so it is very dangerous to separate package loading into several
% different (\input'd) files. It will likely make troubleshooting of errors far
% more difficult. So just load all packages in the inputs/packages.tex file.
%   The exception of course, is package hyperref -- which must always be loaded
% last (see end of the current file).
%
% See http://randomwalk.eu/pages/tex-package-info/ for information on what a
% package does, and information on inter-package problems (e.g. dependencies).
%
% Insofar as possible, I try to only load packages here, and delay the setting
% of configuration options to be done in other places (fonts.tex, style.tex,
% etc.).

\usepackage{caption}

\usepackage{fontspec}

\usepackage{geometry}
\usepackage{graphicx}

\usepackage[bitstream-charter]{mathdesign}
\renewcommand\scdefault{sc}

\usepackage{tocloft}

\usepackage{xurl}



% Set up FONTS to use.

\usepackage{fontspec} % For \setmainfont.
\usepackage[bitstream-charter]{mathdesign} % For math font.

% Customise \mathcal.
\DeclareSymbolFont{usualmathcal}{OMS}{cmsy}{m}{n}
\DeclareSymbolFontAlphabet{\mathcal}{usualmathcal}

% Customise \mathbb.
\DeclareSymbolFont{mpazo}{U}{fplmbb}{m}{n}
\DeclareSymbolFontAlphabet{\mathbb}{mpazo}

\setmainfont[%
  Path=$HOME/.fonts/truetype/ ,
  Extension      = .ttf       ,
  UprightFont    = *-R        ,
  ItalicFont     = *-I        ,
  BoldFont       = *-B        ,
  BoldItalicFont = *-BI       ,
  Scale=0.9 ,
]{CharisSIL}
\defaultfontfeatures[CharisSIL]{Script=latn, Ligatures=TeX}


% Make the page a bit wider. Requires package geometry.
\geometry{top=2.5cm, bottom=2.5cm, left=2.5cm, right=2.5cm}

% Proper format of caption text.
% Requires package caption.
\captionsetup{font=small, labelfont=bf}

% For runin abstract, with a specified margin.
\renewenvironment{abstract}
{\vspace{2.5em}%
  \list{}{
    \setlength{\leftmargin}{1.5cm}%
    \setlength{\rightmargin}{\leftmargin}%
  }\item\relax\par\noindent\textbf{\abstractname.}\ \ignorespaces\small}
{\par\medskip}


% ----------------------------------------------------------
% Use this file for custom things, other than \usepackage's
% (including the \input of other files).
% These are staple commands of my LaTeX diet.
\newcommand*{\emd}{\textemdash}
\newcommand*{\nmd}[1]{\noindent\textbf{#1}}
\newcommand{\nn}[1]{\footnote{#1}}

% ---------------------------------------------------------
% Append new stuff below. Suggestion: begin with \input's.
% ---------------------------------------------------------

% ----------------------------------------------------------

% Name the bib section References.
\renewcommand*{\bibsection}{\section*{References}}

% Use the same font size for bibliography as in footnotes.
\renewcommand*{\bibfont}{\footnotesize}

% Smaller vertical separation between bibtems.
\setlength{\bibsep}{0pt plus 0.5ex}

% Make bib listing with <number><dot>.
\makeatletter
\renewcommand\@biblabel[1]{\textrm{#1.}}
\makeatother

% hyperref et al.
% ***NOTA BENE:*** The hyperref package if used, MUST BE THE LAST ONE included!
%
\providecolors{MidnightBlue}
\usepackage[bookmarks=true,
            citecolor=MidnightBlue,
            colorlinks=true,
            hyperfootnotes=true,
            linkcolor=MidnightBlue,
            linktocpage=true,
            linktoc=all,
            pagebackref=true,
            urlcolor=MidnightBlue]{hyperref}
\renewcommand*{\backref}[1]{}
\renewcommand*{\backrefalt}[4]{%
  \ifcase #1 Not cited.%
    \or Cited on p.~#2.%
    \else%
      \ifnum #1<10 Cited on pp.~#2.%
      \else Cited #3 times.\fi%
  \fi}
