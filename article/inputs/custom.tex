% These are staple commands of my LaTeX diet.
\newcommand*{\emd}{\textemdash}
\newcommand*{\nmd}[1]{\noindent\textbf{#1}}

\newcommand{\nn}[1]{\postnote{#1}}
%
% Shift note number a tad to the left. Allows an optional argument, to
% fine-tune (i.e., correct) the default shift.
\newcommand*{\kk}[1][0]{\kern#1em\kern-0.25em}
%
% Use instead of \nn when preceeded by dot (.) % or comma (,). The optional
% argument corrects the default kerning (shift), cf. previous command. Example:
% \pnn[0.1]{your footnote here...}
\newcommand{\pnn}[2][0]{\kk[#1]\nn{#2}}

\newcommand*{\tsc}{\textsuperscript{,}}

% ---------------------------------------------------------------------------
% Append new stuff below. Suggestions: 1) whenever possible, place these new
% things in a separate file(s); 2) begin with \input'ing those files.
% ---------------------------------------------------------------------------

\input{inputs/custom/mathdelims}
\input{inputs/custom/mathenvs}
\input{inputs/custom/mathgeneric}
